\documentclass[10pt,twoside]{article}
\usepackage{litmag}
\hypersetup{pdfauthor={Oliver Chang}, pdftitle={Ceci n'est pas une revue}}

\begin{document}

\garamond % set content in garamond font

% front cover
\addtocounter{page}{-3}
\thispagestyle{empty}
\begin{textblock*}{8.5in}(-15px,0px)%
\includegraphics[width=5.5in, height=8.5in]{cover.png}
\end{textblock*}
\mbox{}
\phantomsection{}
\newpage

% back of frontcover
\phantomsection{}
\thispagestyle{empty}
\vspace*{\fill}
\noindent Ceci n'est pas une revue\\
National English Honor Society of Gulf High School\\
2011-2012 Literary Magazine\\
\bigskip

\noindent Special thanks to Ms. Ledman and Ms. Winslow.
\bigskip

\noindent Edited by Oliver Chang, Alexander Cranford, Katrina Enoch, Jeffrey Kruse, Elizabeth Mapes, Jeremy Ong, Naser Shareef and Joseph Sylvester
\bigskip

\noindent This magazine can be found online at \url{blankity_blank}.
\newpage

% table of contents
\thispagestyle{empty}
\tableofcontents
\phantomsection{}
\newpage

% begin content
\begin{poetry}{Egocentrism Doesn't Become You}{Brittany Adams}
\begin{flushleft}
Where have you been\\
These last three years?\\
And why did you leave\\
So soon?\\
You promised me\\
That you'd be there\\
For all my\\
Milestones.\\
\end{flushleft}
\begin{flushright}
I have been here,\\
For all this time.\\
Can't you see me?\\
I'm a flash at the\\
Edge of your vision.\\
\end{flushright}
\begin{flushleft}
Left behing to struggle,\\
I know that family waits\\
Just outside me furthest reach,\\
Beyond the hanging tree.\\
\end{flushleft}
\begin{center}
We stand beside you\\
And shout advice.\\
Can't you hear our voices?\\
You shake your head as if\\
The noise disturbs your peace.\\
\end{center}
\begin{flushleft}
I can still see you,\\
With eyes closed tight,\\
Beside our favorite tree.\\
Where the scrap of my dress\\
And the strip of your shirt\\
Still flutter in the breeze.\\
\end{flushleft}
\begin{flushright}
And I remember the very day we\\
First tied our wishes to that tree,\\
Uttering those sincerest promises\\
We too often lost along the way.\\
\end{flushright}
\begin{flushleft}
Winter fast approaches.\\
I stand still and watch it charge.\\
Some days, hoping it will swallow me\\
And leave me with the lost;\\
Others, dreaming of the time\\
Spent warmed by love and anger.\\
\end{flushleft}
\begin{center}
Can't you feel us brush your skin?
We trail our fingers through your\\
Hair---grown long and thicker, now.\\
A cloud of loved ones dances\\
Just one breath away.\\
Why have you forsaken us?
\end{center}
\end{poetry}

\begin{poetry}{Fire}{Katherine Berryman}
\begin{center}
\includegraphics[width=3in, height=2in]{fire.jpg}
\footnotesize{Twilight Fire by William Neil}
\end{center}
Lava spewing down,
Steam rolling across,
Heat against your face,
Light, sweltering air coiling through our finger tips,
Jagged crags piercing the soles of our feet,
Hearing just the wind whistling in our ears,
The sweet whiff of fire and stale air,
But it burns our nostrils.
\bigskip
Just at twilight, a magical time.
As it drifts away,
Steam lifts and
We're left with terra lava
sliding down the rocks.
Taking us away.
\end{poetry}

\begin{poetry}{An Ode to the Forgotten}{Alexis Beucler}
Grandmother drove me in her rundown Sedan alongside the ocean.
She pointed---
``My sweet darlings\ldots{}look at your dear Grandfather. Isn't he beautiful?''
I looked but all I could see was an ocean.
\bigskip
The crystal blue waves flickered with golden licks
\hspace{0.5in} From the all powerful glow of the red sun.
The massive ocean looked as though it would gobble the sun with its immensity.
An immensity which gobbled millions of men---
\hspace{0.5in}Men in battle, men who cannot swim, men who seek for adventure.
\bigskip
Dear Grandfather,
A man who lived inside the sea.
I imagine him in his little (or big for him) sailor suit and cigar in hand,
\hspace{0.5in} Ready to bandage up any member of his crew.
I see him staring out of the peepholes into the civilization of the sea---
\hspace{0.5in} Two-hundred year old sea turtles lazily gaze back.
Tension shoots secret bombs at foreign enemies.
\bigskip
Then I try to recall him on land,
He is nothing but a grumpy old man worn by the sea.
\bigskip
To Grandmother the ocean is her husband.
To me it's an immense mystery of sailor's tails and sprinkled bodies.
To everyone else the ocean may purely just be the ocean.
\bigskip
Before I know it, Grandmother turns away from the water
\hspace{0.5in} And through a golden dilapidated gate overgrown with roses.
Rows and rows of stones of every size were scattered before me.
The immensity of this place betook me for the entrance was so small.
Just as before Grandmother spoke:
\hspace{0.5in} ``Sweet Granddaughter, look at the forgotten.
\hspace{0.5in} One day you will sleep for eternity here or dance in the ocean with 
\hspace{0.5in} your grandfather.''
\bigskip
Each solemn grey tombstone was cracked with weeds chocking it.
Then I thought about the bodies---
\hspace{0.5in} Bodies trapped in cases that want to be freed---
\hspace{0.5in} Free to give their bodies back to Mother Earth.
But instead they are forever forgotten and alone inside of a box.
\bigskip 
At least they were showcased to people who \emph{Remember},
\raggedleft{And now have passed \emph{Forgotten.}}
\end{poetry}

\begin{poetry}{\textit{Woman with a Parasol} Ekphrastic}{Ruhaani Bhula}
\begin{center}
\includegraphics[scale=0.9]{parasol.jpg}
\footnotesize{Woman with a Parasol by Claude Monet}
\end{center}
I wish it would,
\bigskip
strokes upon fat strokes
of dull blues, greens, and yellows
create enlightenment in her
heavy gaze, fragile face.
\bigskip
There is a veil
with each white stroke,
that covers the death in her eyes.
\bigskip
Each strand of white lace
distracts from 
the ill,
depth in here eyes.
\bigskip
The sun shines in the wrong way.
\textit{Moon et aches.}
\bigskip
The son takes the light, and
she falls in the shadow.
\end{poetry}

\begin{poetry}{An ode---Grey}{Nathan Binder}
An ode---
Grey
Writ in blood---
Grey
Hot to the touch, passion---
Grey
When the sky is---
Grey
I feel---
Great
Grey
Gods
We are not the---
In-between
Un-decided
We could be the---
Red martyrs
Yellow insects
Blue champions
Purple outliers
Or gods forbid
Orange
But
Mom, Dad---
We're grey.
It's not a choce, it's genetics.
Maybe I'll---
Grow out of it.
But no---
Grey men have changed the world
A grey man imagined
And a grey man killed him
A grey woman confessed
And it killed her
Great
Grey
Gods
\medskip
%\bigskip
The moral of my ode is this:
beauty is twice beauty
and what is good is doubly good
When you know your color---
Even if it's grey.
\end{poetry}


\begin{poetry}{Six Word Memoirs}{Bobek, Kane, and Salamilao}
\vspace*{0.5in}
A sunny place for shady people.
\hspace{0.75in} ---Lauren Bobek 
\bigskip

The sun descends, books are open.
\hspace{0.75in} ---James Kane
\bigskip

Something's rotten in Denmark. Hamlet's attitude.
\hspace{0.75in} ---Dean Salamilao
\end{poetry}

\begin{poetry}{Brushes With Life}{Caitlyn Borden}
\begin{center}
\textit{
The hallucinations, the non-existent beings
Consider themselves read, swirl in my head,
Screaming, ranting, raving.
Desperate to be heard, they have
No regard for me, and they make me
Crazy, crazy enough to be committed, and
Strive to get help, but without
Leaving them.
In front of this canvas,
Vibrant paints in one hand and brush in another,
I think.
Not like the me before this, not the father everyone thought they know, but like the
``Crazy, crazy enough to be committed'' me.
It's his time to shine, and so I sit back and allow him,
Along with the hallucinations,
To take over and create something beautiful,
Because without the art,
I am nothing more than a man with schizophrenia.
\bigskip
I'm forever drowning in a sea of blue,
Breaking the surface for an instant to catch a breath of air
Before being dragged back down beyond hope.
I gave up a sense of meaning long ago,
Gave up a feeling of home, and
On top of all the names the schoolgirls called me,
I figured all I had left to face was death.
And when it didn't work the first time, I kept trying.
Time after time after time,
An eternity of trying to leave Earth
And its oceans, but all I'd be left with are
Scars.
Scars on my heart, on my psyche,
Scars winding on my wrist like lace.
Before, I had nothing.
Nothing, and no one.
But now I have art.
Pictures of me and the ocean,
Proving that one day I'll be able to swim all on my own
Sans all these doctors.
I've learned I have my entire life ahead of me,
That one day I'll have friends who understand me
Instead of stab my back.
One day I'll have purpose
Where I'll be more than just a teenage girl with depression.
\bigskip
I have no feelings, no empathy,
Only pride at what I've done.
My mother was a tyrant, always in control,
Always oppressing my natural wants and needs.
With no father, I was the man of the house,
The focus of all her attention, 
The outlet she sought when she had ``needs.''
Her reign ended when I slashed her throat,
And it was glorious,
But she never disappeared!
She kept walking,
Stalking me, behind new eyes and new faces,
Annoying me and terrifying me to the point where
I had to kill her,
Again and again and again,
To the point where I had to invent new methods of killing,
Ways I'm especially proud of.
There is no other reason for me to be alive other than to tell my story.
My clock's ticking,
My execution day is coming,
I'm trying to better people like me.
I sketch the modes of murder:
Slitting throats, decapitation,
Dismemberment, suffocation,
Drowning, shooting,
And my personal favorite, stabbing.
Each one shows the life leaving my mother's eyes,
Regardless of the eyes being different I think they're all the same
Hellish woman who tormented me my whole life.
When the government ties me to that chair,
Regardless of my diagnosis,
I'll have left behind something beautiful.
}
\end{center}
\end{poetry}

\begin{poetry}{Question}{Jacqueline Chung}
There is just this one question. There is just this one question.
Just ONE question.
It lingers too often.
\bigskip
I would like to ask any person the same question
This haunts me, pervades me like a festering ant bite that appeared out of nowhere,
The constant scratching at my mind, bleeding for answers,
Leaving scars as reminders.
That one question is the menancing acne on each of your faces that manifests
After every test, after every quiz, after every practice, after every late night study session, after blah, blah, blah.
One after another after another.
Huge. Red. Spot.
It makes you want to hide away, cover it up with the excuse ``they'' call concealer, or whatever the hell it is for you girls.
Guys, good luck to you.
But eventually it will fade away, after all that picking at it,
Leaving scars as reminders.
\bigskip
Sixteen, high school, striving to speak my outspoken mind,
But staying silent.
My mind is like rush hour on I-75.
The cars are like the questions in my head, way too many to count, but they all lead to the same place, or in my mind, the same frustration.
\bigskip
I am Rapunzel, minus the eighty feet of blonde hair.
Trapped and flustered with the same unease and tension in her eyes,
Failing to escape her tower.
\bigskip
Maybe it's just me.
I need to go to a psychiatrist, therapist, or anyone that deals with insanity.
But they don't know me.
Or maybe a friend? Or my mother? My father?
But they don't know me either.
Or maybe that girl in the glass that copies my every more, every flaw, every broken part of me, every lost friendship or trust or respect, every insignificant detail of my body that entails me to believe it's misshapen,
But she says nothing.
Or maybe just someone who will listen? Or give me advice?
But they're just curious to know what's wrong with me.
They don't care.
\bigskip
My question is not just a single question.
It's not a metaphor, allusion, or any type of figurative language.
It pertains to me, selfish or not, I am so sure every single person has experienced
My feeling of unworthiness, that sense of insecurity, loneliness, hopelessness, that emotional roller coaster screaming ``WHO ARE YOU?'' fifteen million times with every twist and turn, spiraling out of control as it cycles over and over and over in my mind.
\bigskip
Who am I? Funny you should ask.
\bigskip
I don't really know.
\end{poetry}

\begin{poetry}{Surfaces of Knowledge}{Alexander Cranford}
The tiles in the school hallways
Whisper to me of students long graduated.
The scuffs of loafers sprinting to class
Or loafers from those dressed to impress
\bigskip
I count the gum stuck to the aging floors.
Impressed, partially, that it has---
So tenaciously---outlasted its chewer.
Who left long ago for products less difficult to digest.
\bigskip
White squares of indeterminate age, 
Paving the pathway to future success or lack thereof.
For many the nondescript pieces of seemingly invincible material
Covered in simple swirls, are not interesting enough.
\bigskip
Rorschach would undoubtedly jump at the opportunity
To delve into the secrets of the grey smudges beneath.
Entire classes follow his example,
Boredom unlocks the philosophical side of anyone.
\bigskip
I've raced down these halls just to sit motionless.
Walked slowly to delay impending doom.
It's remarkable how the insignificant gain value immeasurable,
When the end of something habitual is nigh.
\end{poetry}

\begin{poetry}{Head in the Clouds}{Alexandria Curtwright}
I once deigned to write a poem.
But my thoughts were like wild birds;
They flew\ldots{}
And beat their motley wings
against poetic restraints
of rhyme and meter.
\bigskip
Fruitlessly, I tried to coax them
Into returning to their prison.
But these wild things were averred
By its iron bars,
And its dank and spotted windows,
Which were my eyes.
(They did not let any light in.)
\bigskip
---and they drew even further
into a chimerical Eden.
\bigskip
They roam freely, now.
Along the vast and sun-lit horizons
Of my wandering psyche;
Renderers of idle hands.
(Not unlike wings themselves
with their perpetual trembling.)
\bigskip
Save those rare moments
when they perch
Here, in my yearning bosom
On ephemeral dawns of inspiration.
\end{poetry}

\begin{poetry}{A Lisper's Woes}{Drew Eglin}
\hspace{0.5in} She sells seashells by the seashore
I'm a Slithery Snake.
Can you hear all these S's I'm saying?
Because I can,
\centering{These are a Lisper's woes.}
\raggedright
\bigskip
\hspace{0.5in} When you first hear me 
You get the unsettling sound of static that everyone hates hearing.
All followed by the random acts of spitting. 
As if my mouth was a broken sprinkler head.
\bigskip
\hspace{0.5in} Everyone hates it
Even I, I who lives with it everyday.
For even the creator of the word ``Lisp'' hated us,
For they remind us of our ``Disability'' whenever we explain what is wrong with our speech.
\bigskip
\centering{A Lisper's woes.}
\raggedright
\bigskip
\hspace{0.5in} Whatever I say will never be clean and crisp,
And I'll be haunted for eternity to always have to repeat myself when people can't understand me
\bigskip
\hspace{0.5in} But I've come to a realization: so what if I can't talk like you,
And whatever I say is muffled and skewed.
I am just the same as you, actually no I'm better,
Because I have overcome the speech classes and harassment, mockery, and unkindness towards me.
\bigskip
\hspace{0.5in} I can now say ``Screw you, Lisp, I'll talk my own way!''
The right way, the way god intended.
With the screwed up S's and Ch's, with the sound as if a broken clock was going off.
Because you know in my eyes, I'm not the one with the impediment. 
You and everyone else are.
Because I can sch my S's and sch my Ch's, and you can't.
\bigskip
\hspace{0.5in} I have the gift to torment those I hate.
I can spit on them with the slightest of lisp, and blame it on it.
I can mess up reading a word's pronunciation, and go ``Damn, it's all the lisps's fault.''
\bigskip
\hspace{0.5in} So I guess this isn't a Lisper's Woes.
It's a realization.
A Lisper's realization, that everything's okay, and I'm better than all of you.
\end{poetry}

\begin{poetry}{I'm Listening (Perhaps That's the Problem)}{Katrina Enoch}
shut up.
just
\hspace{0.5in} shut
\hspace{0.5in} up.
\bigskip
She assigns twenty pages of homework
As if she was clever.
(She's too enthusiastic about this.)
Twenty pages of Literature, perhaps?
``Oh, you should be able to do that!''
I write yet another assignment in my planner.
(I'd rather watch \textit{The Golden Girls}.)
\bigskip
Fifty derivatives are to be calculated
By the work of magical chemicals
Within my brain.
(Synaptic gaps and neurotransmitters!)
Redox reactions are to be balanced.
(The joys of OILRIG.)
Who was Jefferson's Secretary of State?
(Just put Johnson. It's a common political name.
Perhaps he won't notice.)
Question word, est-ce que, subject, verb.
Question word, Johnson, the limit as the redox approaches a state\ldots{}
\bigskip
Quick! Mount Saint Helens has erupted!
The brain is now mush!
\bigskip
No! I am not Einstein, nor was meant to be!
\bigskip
In the room women come and go, talking of Michelangelo.
(Remember, twenty pages of Lit.)
\bigskip
shut up.
just
\hspace{0.5in} shut
\hspace{0.5in} up.
\bigskip
``The exam grades were not impressive.''
The sorrow felt by my teacher does not touch me.
I pride myself with the shield of a 98.
I show my mother, giddy as a schoolgirl,
``Mommy, mommy! I got a 98 on my exam!''
\hspace{0.5in} ``Your sister got an 80 on her College Algebra exam! Isn't that great?''
``Did I mention it was on my Calculus exam?''
\hspace{0.5in} ``A 98, you say? That could have been a hundred. Why can't you be
 \hspace{0.5in} like your sister?''
\hspace{1in}``Katrina, I have something to show you,''
Beckons my father, I know exactly what it's concerning.
``Dad, I really am not interested in petrol---''
\hspace{0.5in} ``Princeton has a highly recommended Petroleum Engineering 
\hspace{0.5in} program.
\hspace{0.5in} They make seventy grand straight out of college.
\hspace{0.5in} What would you do with film?''
\bigskip
shut up.
just
\hspace{0.5in} shut
\hspace{0.5in} up.
\bigskip
I cannot organize my thoughts to be cohesive.
Quick! Who is the father of modern chemistry?
No! I shall not stand for this! There aren't enough hours in a day!
I study for several hours a night, and attempt whatever homework I can complete
With hopes that I will not drool upon the page!
Club meetings occur everyday
For an hour at least.
Add the ridiculous amount of time my mother
Wastes before getting in the car to pick me up.
(I really need a car.)
\bigskip
``In New York, you couldn't get your license until you were eighteen.''
\hspace{0.5in} ``What relevance does that have, mom?''
``Well, the same goes for you.''
\bigskip
shut up.
just
\hspace{0.5in} shut
\hspace{0.5in} up.
\bigskip
It's eleven already?
If so, I'll finish the rest during lunch,
For it is time for the highlight of my day\ldots{}
Sleep: My only escape, if only for a few hours.
\bigskip
However,
I cannot sleep.
I toss, turn, and grumble here and there.
(It's got to be one in the morning. That leaves me with five hours left to sleep.)
I open my eyes to confirm my suspicion.
\bigskip
Voil\`a!
\bigskip
I find my homework next to my pillow,
My mother clutching my sister's College Algebra test,
My father with an Exxon Mobile sign,
My teachers with slashed papers,
(Their pens are knives.)
I see Einstein, T.S. Eliot, Betty White,
Antoine Lavoisier, and James Madison,
Have all gathered together,
Piled upon my bed.
\bigskip
``What are you all doing here?''
I scream.
My thoughts collide into each other
Faster than gas particles inside a balloon.
\bigskip
``Hush, Katrina,'' they say,
``We're trying to sleep.''
\end{poetry}

\begin{poetry}{\textit{Untitled}}{Brittany Gervasi}
Whatever, Grandma, I don't even care!
He annoys me, he eats my candy.
He steals all Mommy and Daddy's attention.
He's stupid, mean and stupid.
Everyone focuses on Ty this, Ty that.
Tyler, Tyler, Tyler
What about me?
I'm cute too!
My friends even colored pictured and cards for him just 'cuz his face turned all funny and he talks weird
He feels kinda icky, but he'll get over it.
We would be better off without him anyways.
\bigskip
After that night, life tasted rich
Like vanillla ice cream, his favorite.
We raced motor bikes on virtual gravel roads until the horizon swallowed the sun.
Our blanket forts, fully equipped with Sega controllers, frosting cans, pills---
We were unstoppable! Set for life!
Conquerors of carpet between beds and dressers. Underwear our armor, 
inedible food, our neighbors brought as ``gifts'' (surely to weaken our defenses) as our weapons,
Ready for battle, we knew we could take on whatever dared to attack.
But, cancer doesn't like to be challenged.
\bigskip
Mom and Dad cradled Tyler out of All Kiddies not even seven days after he found his way in.
Little Monster crawled from room to room heading straight towards a Fischer-Price table
Ruined my sun catcher he did, just as he ruined everything for me;
Dumping truckloads of glitter over the white and crimson.
Of course, nobody cared that he murdered my dragonfly.
Just me, Problem Number One.
Though they never misconstrued Perfect Child, Child Number Two,
as anything less that splendid.
Only I knew him as the brat who still got bedtime stories.
My old bedtime stories.
\bigskip
I pleaded for someone to wake me if it happened.
Not once did thoughts ever wander into my thick skull that it could even transpire.
Especially not to him.
7:09 A.M.
I only said, ``Oh, okay\ldots{}Um\ldots{}I'm going to my room now, bye bye, Ty.
I love you.''
My biggest nuisance. My accomplice in crime. My groupie. My prot\'eg\'e.
Gone.
Compassion finally broke the dam.
They came without restraint as they shook my small frame, salt stung my cheeks
as they looked to form puddles of regret on a people pair of footsy pajamas.
Karma, why did you pick now for revenge?
His transcendence to Heaven transformed my life into Hell.
\end{poetry}

\begin{verseEnt}{Pictures}{Kendra Jones}
\lettrine{``O}{nly} five more minutes,'' Will thought. ``Then I'm out of this Hell-hole.''

As a paper plane whizzed by his head he decided to tune back into his history teacher's tangent. Something about Romans, like he cared. He scanned the room and noticed everyone's dull face. It was Friday after all. He looked out the window and begain to tune his teacher out again. A cool draft entered into the room bringing about thoughts of how autumn was approaching fast. He was jolted back into reality with the harsh sound of the dismissal bell. He leaped out of his seat not bothering to hear the homework assignment. He wasn't going to do it over the weekend anyway. No point in wasting any more time in this prison, he reasoned.

After making his way through congested halls, he finally reached outside. He bounded down the steps and was embraced by a cool breeze. These are the days he enjoyed when he had to walk home. As he walked along on the sidewalk, a gentle wind enveloped him carrying leaves and a piece of paper. He eagerly followed it until it got caught in a chain link fence. He plucked it from where it laid and turned it over to reveal the most stunning girl he had ever seen with her perfect smile and eyes shining like she had a big secret. She was holding up two fingers, the Peace sign, in the cutest way possible. Forget Savannah from last year, whom he was totally infatuated with. This girl was out of this world and he had to find her. No. Had to have her.

That night he laid in bed, thoughts thrashing around his head like a ship caught in a storm. All for that girl. When he finally reached the eye of the storm, all thoughts ceased and the ship was safe. He heard tapping on his window. Hesitant but unrelenting. Making sure he wasn't in a comatose state he waited for a minute. When it continued he decided to get up and open the blinds. It took him a minute to realize who was there, standing behind his window. It was that girl. The girl from the picture! He gasped at the simplicity of her beauty. Underneath the full moon her eyes shone brighter that the Sun; her hair paler than a ghost. She was pure poetry. His thoughts were interrupted by the softest of giggles. He focused back on her in time to see that she was gesturing for him to follow her. He silently slipped out of this window and walked, aimlessly, following her into the woods across the street. He was halfway across the street when bright lights slammed him back into reality. A horn honked, and like a startled deer, he looked at the car in front of him. Incoherent to the profanities being shouted at him, he sullenly walked back to his house wondering why he was out in the street in the first place.

For the rest of the weekend there were no incidents with the girl. No awakening in the middle of streets. No\ldots{}nothing. Just him staring endlessly at ths girl in the photo. By Monday morning he was a wreck, suffering from insomnia over fantasizing about a girl he wasn't even sure existed. He looked hopefully around his school and asked his friends, even showed them her photo. All his attempts proved fruitless. He was slowly going insane, slipping away from reality by this mysterious siren. All friends lost contact; he grew distant from family. By next Monday night, he was a different person. His body, soul, and will to live were dedicated to this girl. And on Wednesday night, as he lay down in bed for another night of no sleep, he heard tapping on his window, persistent and urgent. He sprang out of bed, rushed to the window, and ripped open the blinds. And there she was, his muse. His siren. His love. She let out a giggle and beckoned for him. He followed without hesitation, crossed the street and into the woods all underneath the dark sky. 
\begin{center}
\line(1,0){200}
\end{center}

For the next week, search parties combed the woods to find Will before officials declared him missing. Everyone wondered what drove him to disappear, but they all agreed that his mental state had been declining over the past few weeks. The next day at the end of school, students held a vigil for Will outside and prayed for his safe return. As everyone left one by one, a boy named Chad decided to stay behind. he wanted his good-bye to be in private without the accompaniment of the entire student body. As he walked over to the vigil, a nice autumn breeze went by, and along with it a piece of paper. It landed at the base of Chad's feet. He stooped down, picked it up, and turned it over. And to his surprise he saw the most attractive girl in the entire world with her flawless smile and stunning eyes. And oddly enough, she was holding up three fingers\ldots{}
\end{verseEnt}

\begin{poetry}{The Evolution of Man}{Kristina Kanaan}
I am here, waiting. 
Time that is so beautiful and infinite,
Time that is finite and replaceable.
Speak to me in mediums in which I cannot hear.
make me feel your reality.
\bigskip
Center wheel, dial, second hand, face.
Build a clock evolving from the sun,
feel the sun fall below the horizon and into oblivion.
Precious time must be kept in calculated manners.
Time, given to be wasted,
expires like milk without a proper burial.
The second hand stops ticking,
and you do nothing.
I am still here wasting.
\bigskip
A force not to be reckoned with.
an entity gone untouched, unprovoked.
\bigskip
Today, I call you out.
dared, you coward, do your worst!
Remind me why you control my life and my death.
Why you control the split seconds that send mothers to graves,
and second chance children with seconds left to spare.
Why you greedily hold onto time, giving Shakespearean star-crossed lovers years too little.
\bigskip
Stop hiding behind shiny glass faces and golden frame,
you are nothing until I am proved wrong.
Make me feel you.
Make me regret ever cursing your name.
Make me bleed the springs that wind your clock.
\bigskip
Make me \underline{fear} you again.
\bigskip
Live time.
Bleed blood of seconds,
sleep in pools of it, with countless spent hours.
Time is left wasting, until the body is drained of it.
Feel every second of every minute pass by,
like the rush of a frenzy.
Everyone is running to make the time last.
As I wait, every pore bleeds out the blood of time,
with a rhythmic ticking sounds as it hits infinite soul floor.
Tick tock, drip drip, tick tock.
Woozy and waiting a little too long.
Time's up.
I am no longer waiting.
\end{poetry}

\begin{poetry}{Ode to a Longboard}{Michael Keller}
Crafted miles away,
fell into my unworthy hands.
Bamboo created a royal aura,
The smooth lavendar wheels spin.
My feet were peasants until 
The gracious board accepted them.
We soared down the road like a bird in the open air,
Floating whichever way the wind blows.
Together we rule our kingdom,
Escaping reality.
We become an artist,
The road our canvas.
The picture painted for all to see.
My spaceship taking us,
To our own planet.
Far away but still at home.
\end{poetry}

\begin{poetry}{Ode to Coffee}{Calee King}
Drink the stress away,
With the intoxicating smell lingering for  a while, won't you stay?
Listening to the sound that's\ldots{}
Dripping, dripping, dripping.
With the power to extract the fatigue from my fragile body
I can feel my spirit strengthening second by second.
But the serenity is only temporary,
And like a time bomb ticking away
I eventually explode into an abundance of energy,
A spontaneous burst of flame.
Bouncing off the walls,
Unable to sleep,
I am indestructible, accomplishing tasks at an unnatural speed.
With this addicting legal drug
I am Superman (or woman, that is),
Flying through my homework,
My eyes glued open with uncanny alertness.
It is the very key to my existence,
The reason I am able to complete my abundant schedule;
For without coffee I would be a mere mortal,
Unable to overcome my lethargy,
My kryptonite.
\end{poetry}

\begin{poetry}{Ode to Hunting}{Carl Krondahl}
From the origins of man,
It lays at its most archaic
From Bows, to Muzzleloaders,
To High power rifles and shotguns.
Some find it a disgrace,
Others find it majestic and relaxing
This taboo religion in some societies
Is found home in the backwoods of the world
From big ol' bucks to snapping gators,
And from mean grizzlies to smallest of birds.
Most who take interest in this taboo activity
Are often that of the farms and hills in America.
From those who drive old pickups to those who drive new ones
Some drive cars others drive semis
Some play the Banjo, others cant
This taboo religion is one of the South and the Hills
All know how to ruff it
Whether it be building a fire or getting food
They know how to survive and build their own bed
They don't buy unnecessary things like fancy clothes
They just use what they got and take what they get
They know what's right and what's wrong
And whether it's a Matthew's bow, Traditions muzzleloader,
Remington shotgun, or Winchester rifle.
It don't matter where you're from
If you try it once, you're hooked
The food is good eatin' and 
As my granddaddy always said Invest in shotgun and rifle rounds
And when food prices go up, ya gotta find ya own.
\end{poetry}

\begin{poetry}{Ode to Dreams}{Philip Kubiszyn}
Rich in grime and filth,
a child grasps doubloons and pearls
whilst thrusting his sword---
a gut-wrenching roar.
Island breezes drag stench
like scallywags to a decrepit chest.
Lame is the heart of Black Beard,
bound by golden rope,
struck with envy at emeralds
from victorious plunder
\bigskip
There sits a man, 
red leather embraces his frame
as he caresses his lover's curves,
invigorating adrenaline fuels his passion.
A cloak of elegant jet
accelerates his lust,
overwhelms senses
with potent sensual beauty.
Unsullied Italian rose,
devoid of thorns of imperfections,
handled by tender touch
in his Ferrari of jet.
\bigskip
There rests an elder,
perched upon a mountain,
his very soul floating in the sun's domain.
He watches majestic doves flutter by,
his tranquil wife grinning
as she admires her husband's awe
with tear filled oceans of blue.
The elder looks down into the valley
and observes a man with his Italian lover.
He looks to the sea and observes a young boy
soiled with riches and dirt.
Darkness falls.
That which can never last
slowly slips away.
\bigskip
But darkness never hinders the elder,
his soul still floating in the sun's domain,
able to embrace his wife at last.
\end{poetry}

\begin{poetry}{November}{Savannah Law}
Our love is made of
sand \& soot
and gritty things
like industrial sunsets
on glass bottle
beaches
and subtleties---
whiskey tumblers
with lipstick
stains,
mentholated ashes forming 
death pyres
in your Prada pumps---
and of soft
feathery malignancies
like shooting stars
and the dust of bones---
of things one cannot
place upon
a finger,
like strobe light
memories of
sober hands
with a drunken guise
and a girl with a
waiting
smile.
\end{poetry}

\begin{poetry}{You Dita Von Tease Me}{Chelsi Mackin}
\begin{center}
Doll, you've got me balled up,
Why you've got the hose running and the ink blot running amuck?
You must be off in that fuzzy world of women again,
By Golly, those cherry red lips really do rev my engine,
And those symmetrical arches raised high above your surprised expression
I'm sorry I startled you, Baby,
But your glossy locks drive me crazy,
That cigarette smile coaxes open my pack of Lucky Strikes,
Exhaling each puff to the sky,
I saw you in the stars tonight.
The angels tried to tack up a celestial portrait,
But the beauty mark wasn't quite right and the hips were too sharp.
\end{center}
\end{poetry}

\begin{poetry}{The Slug}{Mouzel Manugas}
I watch you slither
In your transparent slime.
You slide so slowly.
I scrutinize every motion
Of your muscular foot
On the pavement.
\bigskip
I grow envious.
Living second by second,
You don't even gaze
Ahead at the stick
Blocking your path until
It touches your feelers.
\bigskip
Guided by instinct
You finally reach a 
Pile of dead leaves.
There you lie,
On the target you
Tirelessly inched to reach.
Living only a slick
Glossy line behind.
\end{poetry}

\begin{poetry}{Among the Heaps}{Nicole Manugas}
Incandescence, candoluminescence,
the lime light which shines on you and I
reveals our lazy afternoons spent daydreaming,
smoking, sitting, drinking heaps
of uselsss beings.
Inert, idle, feckless piles of trash
with hands and legs and hearts made of
discarded soda cans, worn out old shoes,
and Tuesday's left overs from around the world
\bigskip
Symbols of scum, and dirt, and pointless existences.
We project beauty in our unmoving, lazy ways;
in our carefree frozen forms,
in the outline of our lighthearted silhouettes.
Back to back, you and I,
with the wine that never leaves my cup,
and the smoke suspended in the air.
\end{poetry}

\begin{poetry}{Isolation}{Jerad McMickle}
Turtle doves cast shadows over a lakebed.
A school of otters form a solidified squeak
as they send streaks of white across the 
light blue river floor.
Bird talons twitch from the delight
that accompanies the freedom to be
able to drop and pluck greedy fish
who stagnate in the shallow waters.
Small footprints shimmer from fresh dew
indented by chipmunks stalking hazelnuts.
Their carefree desire to follow nature
brings memories of the thoughts of man.
I stand alone with the two
inquisitive eyes of desire.
\end{poetry}

\begin{poetry}{What's Important?}{Jon Morgan}
\begin{center}
You say I'm quiet and I need to talk more.
What for?
Do you want to hear more about me?
What I've been through?
It's not important.
\bigskip
When I was two
I was less active than the untouched toys in my crib,
But the pediatrician said, ``He's fine, come back in a week if you're REALLY worried.''
``Something's wrong with my son,'' my mom urged.
Scans showed nothing, except white blood cells far off the charts.
So they gave me a spinal tap, but I didn't cry,
Unlike when I kicked and screamed over a sponge bath moments later.
A shot everyday for a week.
Relearning to crawl, relearning to walk, relearning to talk.
Eight years of speech therapy,
A lifetime of readjusting, dealing with the irritations of hearing aids.
But I was just two, these are mere stories to me.
They're not important.
\bigskip
And when I was seven,
A bulge grew on my neck,
Again they said I was fine, nothing to worry about.
It swelled to twice its size,
Waking me in the middle of the night, sharp paints running down my neck.
The next morning, I was paler than the wrist bands the nurses put on my arm.
Diagnosed with Stage 1 Lymphoblastic Lymphoma.
Prescribed many pills to swallow:
A steroid a day,
an appetite pill each day,
seven yellow pills every week,
a big gel pill once a week too,
But worst was the Prednisone each and every day.
I couldn't swallow them, so I had to chew
And taste that awful taste,
A taste that overpowered anything we tried to hide it with,
A taste that couldn't be scrubbed out with toothpaste,
A taste that I had to taste again if I couldn't stand it and threw it up,
A taste that taught me how to swallow,
So I no longer had to suffer from it.
These pills helped me to grow to twice my size,
And the chemo stole my bright, blond hair.
Bald and big and lacking energy to throw a ball or walk up stairs.
Forced to wear a white mask in public,
Appointments every week for years,
But I was just seven, these are vague memories to me.
They're not important.
\bigskip
When I was sixteen,
There was a pain that caused me to vomit everything I ate and drank,
But the doctors said I was fine, Stomach Flu.
The meds didn't help.
Still hunched over in pain,
We went in for scans three times,
All inclusive.
They signed me up for exploratory surgery,
And shoved a tube up my nose and down my throat,
Causing me to gag and vomit all over myself.
A ruptured appendix this time.
I woke up from my surgery
With a catheter, an IV in my arm, the tube up my nose, and bulb off my side.
Bed ridden, too sore to walk.
Unable to eat during the week of Thanksgiving.
After some rest they removed the tubes, and went to remove the bulb:
``Will it hurt?''
``Yes.''
It was a strange and unusual pain, a cord yanked from my side.
I returned to a large pile of make-up work and mid-terms.
But I'm okay,
It's wasn't important.
\bigskip
Four scars are all that remain,
The one on my neck, two on my chest, and one down by my stomach.
But I'm fine,
It's not important.
I'm not the only one who sat up all night, wondering if her son would be alright.
I'm not the one who retired early to help her grandson get to the hospital every week.
I'm not the anonymous who paid some random sickly family's electric bill.
I'm not the one who went door to door and shop to shop raising money for a fellow classmate.
I'm not the one who set up a fund raiser, raising thousands of dollars for a neighborhood kid.
They're the inspirations.
They're the ones who are important to me.
\end{center}
\end{poetry}

\begin{poetry}{America Youth}{Corey Palermo}
If I knew the center of my universe
I would stand there and recite poetry to my children,
In the hopes that my verse would traverse the lockers
Of granite and lead, perpetrated by a collective sense of
Nationalist, Foundationalist, Indoctrinationism
\bigskip
The kind of relentless corporational conformitism
Best befitting halls of cubicles---
God's free-spirited attempt at a fa\c{c}ade of free will
\bigskip
If my universe had a center, then I would be called Father
And My children would not be lost to a pitiable piety
Of a superficially religious society.
\bigskip
But I am not the Father, not am I the Son.
I am the shadow of a neo-platonic shadow
Set by a desecrated Holy Ghost
\bigskip
I am less than His shadow.
I am a pebble, shaped by earth and cleaved by
Dense minerals.
\bigskip
I feel green\ldots{}
\bigskip
I am the stone-cold secular emerald at the back of the room.
If ``Brains'' is the mind's product, I come before it,
Demanding rubies from a reality of triviality
In a room lined with coal.
\bigskip
Onyx is my closest semblance
To a mutually-cleaved resemblance
To a history---of being cleaved by coal
\bigskip
Watch us as we watch zirconium at the front of the room,
The symbolic emulation of a popular degradation of spirit---
A mundane sociological destruction of metaphysics,
Sold down through the generations of rocks to the highest bidder.
\bigskip
Yes, we are the semi-precious stones in a red-white-and-blue non-spiritual room
Lined with coal, and infected by a non-prospective consensus to an evangelical disease.
We nod our cleaved hands in a desperate attempt to stay-asleep.
\bigskip
Onyx, the victim of an all-loving hatred,
Sent to the back of the class because he cannot be left behind
He's tied behind to the rugged rucksack tied to the crooked back
Of an emerald.
\bigskip
They tie him tightly so he won't drop out.
\bigskip
We are the semi-precious stones
Lost in a realm of mineral mines, lined with mines,
Rejected by the diamonds of this day.
We are the broken ceramic place below the Tea Party's table---
Watch us as we drink gallons of pyrite,
Victims of a semi-contrite executive conglomerate of stone-cleavers.
\bigskip
Watch us watch diamonds turn to coal before closed eyes---
The products of a blemished embezzlement of a
Mine-riddled-mine-rid-of-mines shaft,
Filled to the tip with semi-precious stones, and empty of pyrite---
Drank by US, by a futureless future of gems---
The clones of a zirconium lifestyle.
\end{poetry}

\begin{poetry}{Ode to My Paper}{Devon Ritter}
I had bought
A pack of paper,
Down at the local store.
I had kept it neatly tucked away,
A favorite childhood plaything;
As if to protect my dead tree,
And protect it like a treasure,
Containing the finest gold and jewels.
But as one day should have it,
I needed a piece of my treasure.
I tore carefully at the sticky plastic,
A surgeon operating.
The crackling and tearing as if flesh being removed from the body.
As I looked at the paper.
This internal working of the body,
I took the first, crisp piece out.
Ideas swirled in my mind,
A surgeon with a thousand ways to complete his procedure.
Perhaps this will be the birth place
Of a story
Of a drawing
Of a school assignment
Of sheer boredom.
Or perhaps it will be a place of death.
A failed attempt that will result in the early retirement of this precious paper.
Or of writer's block,
And the sheet will forever remain in a comatose state,
Waiting to be resuscitated.
\bigskip
As I gently set the sheet down,
A sharp pain. A papercut.
The tree's last revenge.
And I wonder how something as fragile as paper,
Could leave such a small, painful wound.
I begin to see a woodcutter,
Making tiny little wounds on a tree.
Slowly, slowly, until it crashes down.
Will that be me? Little wounds,
Slowly, slowly, until I collapse?
\bigskip
I look at my result.
This piece that I stitched together with love and care,
That I nursed and nurtured through its course.
Like a surgeon, I step away from the table,
Satisfied with my work.
And on I go to the next sheet of paper.
My next treasure piece,
My next patient for the surgeon's table.
Ready for the cold edge of my pencil,
My knife.
\end{poetry}


\begin{poetry}{\textit{Untitled Slam! Poem}}{Nicole Payne}
Sometimes I feel that life is one big hippie train,
colors everywhere,
feet jumping through the dry air like jackrabbits,
arms flinging around each other, with warm, affectionate antennae.
Our hippie train keeps ``coo-coo-cachooing,'' wah-hooing, massively growing
and there's old Johnny in the way back,
beating down on his sandy drum
BUM BUM BUM!
It takes me back, yeah, back to the days 
when we trudged through the desert heat
in search of just a bite to eat,
on our New Days journey across America.
Back Beat Bearfoot kept on smoking, smoking, puffing, toking
that good stuff,
keeping everyone calm and ready,
even when our paranoia spoke, oh so loud,
into our faces,
HELLO THERE AGAIN! JUST BACK TO SCARE YOU!
Even when the brown bears knocked over our tent,
even when our brain cells went
KA-POOSH! Out of our heads
and my lovie kept singing,
and my lovie kept singing,
``Never turn around because you ain't turning back\ldots{}''
\bigskip
Sometimes I feel that life is one big orgy;
creatures copulating,
eight at a time.
enjoying the simple pleasures,
indulging in the sinful stuff
that they call bad.
But we don't care, we are just glad
to be open and tested
out on each other.
And the very hairy man stepped on a tack,
screamed out ``GAHH!''
While he was doing strange things in the back
of a caress-loving woman with hardcore curves,
directing everyone in the right direction.
And I was there,
right next to my love,
but we were just watching;
didn't want to take the risk
of catching a crab
or groping a bad muffin.
So we dropped acid
and let the music enter us
and my lovie kept singing,
and my lovie kept singing,
``Never turn around because you ain't turning back\ldots{}''
\bigskip
Sometimes I feel that life is a labyrinth,
confusing me and using me
to make a fool out of myself,
just the way I did when I grabbed your mother's rear as she bent over the counter because I
thought she was my girlfriend.
And then I turn down a wrong path,
somewhere in the mysterious labyrinth,
thinking I am going to die
because I am all alone
and I dream for a second of my hippie friends,
how they took care of me,
held me in their arms at night, telling me of their sentimental secrets,
being together and feeling like it was never going to end
and it flashes before me
and I cry, scream, and panic.
It's a bad trip, man, it's bad!
AM I GONNA DIE?!
``Yes, you are,''
says the cockroach clinging to my foot.
I'm all alone.
No one to comfort me.
No one to save me.
I had them back then,
but I took a wrong turn, it was a dumbass move.
Shoulda stuck with them\ldots{}
Dead.
And my lovie kept singing,
and my lovie kept singing outside,
``Don't turn around because you ain't turning back now\ldots{}''
\end{poetry}

\begin{poetry}{Ode to Bubbles}{Brittany Reid}
Iridescent bubbles leave the wand as if by magic;
the offspring of Iris and the sun,
gleaming.
Eyes that glint with the secrecy of the message they carry
for the slippery eyes of the gods alone.
\bigskip
Floating towards the heavens like souls
that slip away as feet skid on soapy floors,
as small children slither down a slip n' slide
ignoring the rocks and twigs that pierce them.
Ascending, though their fate is cloudy,
polluted with their sins and the uncertainty of forgiveness.
Nevertheless their beauty is unmarred,
as they are made clean
by that which makes them rebel
against gravity's constraints.
\bigskip
They drift skywards
until they are lost in the expanse.
Those below are left to marvel
at the messengers that know so much
and are about to learn the answer to man's most prevalent question.
Whether they reach their destination is uncertain.
Perhaps they never do;
perhaps they burst.
\end{poetry}

\begin{verseEnt}{20/20}{Olga Saniukovich}
\lettrine{I}{gor's} head bumped against the thick glass as the wheezing monster swayed along the tracks. He scowled and rubbed his forehead, red and sweaty from sleeping in the heat of the ancient train. He sat up on the stiff wooden seat and grimaced; he must look like an absolute mess, he thought. The bursting scenery of God's rolling Creation was lost on his bleary brown eyes. Why was he making this journey? He remembered the confusion on his wife's face when he had brought the telegram in to the kitchen. ``I have to go back to Domoy for the weekend,'' he had growled. ``Purely business matters, my dear, I won't be held up for too long.''

He hadn't seen Domoy in years. Not that it mattered to Igor; he had thought he'd left it behind for good\ldots{} Every other week he numbly crumpled up another unopened envelope with the town's name scripted on the return seal---but that was it. He had boxed up everything he had once owned that bore the town's name in its memory and stowed it all in the attic, where it now lay covered with a satisfied layer of brown dust. Why had the telegram had to come? Igor wondered. He felt a bitter taste in his mouth. He grimaced and turned his face away from the shining sun.

He produced a wrinkled white piece of parchment paper from the pocket of his slacks and re-read the first lines of the telegram again:

\texttt{Mr. Igor Nelyubov---}

\texttt{We regret to inform you that your mother, Constance}

\texttt{Nelyubova, has passed away\ldots{}}

His suddenly sweaty palms softened the paper. When he had first heard the words echo in his mind, he had felt relief surge through his body. He had thrown away the last of the persistent, unopened letters. His memories of his life before Gnill could finally be packed away in their own airtight boxes in the far corners of his mind, and he could finally be free\ldots{}

Igor glanced around the cabin. The other passengers only briefly returned his gaze before burrowing back into their coats and newspapers. Nobody had seemed to catch an idea of the contents of the telegram---then again, the people from Gnill were different. Everybody lived on their own, white picket fences enclosing patches of brown lawn, curt politeness floating quietly through cracked streets. Igor stuffed the paper back into his pants pocket, regretting having boarded the train destined for Domoy. He squeezed his eyes shut, growing ever more annoyed with the persistence of the bright ball of light outside the dirt-streaked window.

He heard laughter. The kind of laughter that erupts from a spoiled child's mouth, accompanied with gleaming eyes and a pointing finger. There were so many, oh God there were so many of them. They surrounded him in the courtyard, shoved him onto the stinging gravel and shouted, kicking his carefully ironed uniform with dusty shoes. Over and over again, it seemed like they would never stop---Your mama's a witch! Your mama's a witch! He rolled over onto his side, hugging his bruising head with weak arms, dreading the blows and wailing for them to stop. Stop, stop, stop, he cried. The laughter rung in his ears, the pangs in his side throbbed. Over and over again, Your mama's a witch! Igor's mama is a witch!

Igor's eyes flew open. He was breathing heavily, shirt soaked with sweat. Your mama's a witch! Your mama's a witch! He shook his head violently to force the laughter out. A lady with a rolling cart stood next to his seat, eying him quizzically. ``Something to eat, sir?'' she ventured. Igor pulled a crumpled bill out of his leather wallet and grabbed a few unnamed cookies and asked for some whiskey. The girl placed the bill in a metal box and replied coolly, ``We have wine.'' She poured him a plastic cup-full of lukewarm white wine and rolled her cart past him, one wobbly wheel twisting and jerking, straining the woman's arm muscles as she steered the beast onward, never once dropping her composure.

Igor sipped the wine in an effort to calm his frazzled nerves. The liquid stung his mouth and he spat it back into the cup. It tasted sour, almost like vinegar; inspecting the cup closely, Igor saw that the contents were yellowing undeniably. This irritated him even more---whiskey was what he needed\ldots{}

He pulled an old handkerchief from his shirt pocket and wiped his moist brow. He paused and admired the stitched pattern on it---his daughter had made it for him years ago, when she was just a girl. It depicted a crooked family---Igor, his wife, his daughter and son---in the large, clumsy stitches of an impatient child. On the back corner of the handkerchief she had added the words: ``I love Daddy'' and given it to him for his birthday.

Igor felt a pang of bitterness as he ran his calloused thumb over the letters. \emph{Daddy}. He remembered asking his mother when he was a child, Where's Daddy? When is Daddy coming home? as he sat by the window watching the sky gradually turn grey and the night descend like a thick velvet curtain. His eyes searched the lonely street, waiting for that familiar man to turn around the corner and come strolling up towards the chipped door. Daddy's not coming home, his mother had told him as she wiped her tired hands on her apron. Daddy isn't coming home anymore, sweetheart\ldots{} Igor's throat grew tight and tears prickled against the backs of his eyes. He felt his mother place a warm hand on his shoulder and he jerked away. You made him go away! He had shouted. You made Daddy go away!

The stinging in his eyes blurred his vision and he crumpled the worn handkerchief in his clammy fist. He had no recollection of his father---he had always come home late, when the streetlights had long since cast a warm glow on the sidewalks, and grunted something briefly to him and his mother before going to bed, only to rise before the sun and slip past the door early the next morning\ldots{}

Igor watched the trees fly past to the rhythm of the train's \emph{ka-thunk ka-thunk\ldots{}ka-thunk ka-thunk}. He had had a miserable childhood. Divorce was unacceptable in the small town of Domoy, and impossible to keep a secret. Though his mother made a great effort to keep a warm smile on her scarred face, all of the gossiping old women found out within a week that her husband had left her. The news spread like wildfire, and Igor felt the judgmental gazes of the townspeople bore into his skull day by day as they studied him and his mother walking down the bustling street. He hunched while he tottered along, always at a distance from his relentless mother. She kept trying to pull him closer to her, whispering that they had to stick together. Igor pulled away and ran on his own. He didn't want to be seen with this woman any more than he had to; she was ruining his life enough as it was. His almost daily beatings at school cemented the hatred more and more in his mind, until he was calling her a witch in his mind as well\ldots{}

She had taken the job at his school cafeteria to try to be closer to him; and his classmates never let him alone for a minute. He dreaded the resounding echo of the bell at noon that beckoned his entire school to lunchtime. He felt everyone sneaking glances at him out of the corner of their eyes as he turned his head away from the woman who gently placed a bowl of hot soup in his hands. I love you, Igor, she had called, every day---even though he was sure she saw the way the kids sniggered and poked at him as he shuffled quickly and silently away with his head hung low. Igor grimaced, the memories taking over. No, that was just it---she didn't see. She couldn't; she refused to see. The space where her right eye should have been was a hollow, a flap of skin sewn over and scarred, so that she could only peer at Igor with her left eye, incapable of seeing the pain that she caused him with her disgusting, shameful presence in the midst of his peers. He walked over to an empty table and ate his warm mush, alone, head in hand; his entire childhood and adolescence he spent miserable, lonely, and ashamed. He couldn't fight back against the pain the other kids brought him, so he fought back against his mother. He pushed her away and ignored every attempt she made to invade and poison his life any further\ldots{}

The day he had received his acceptance letter to the prestigious university on the other side of the country was the happiest day of his life. He floated on air and went out to celebrate with a glass of champagne at the bar---under the wary eye of the public and the nervous hand of the bartender. He downed the alcohol, alone, his mood slightly dampened by the flatness of the beverage and the unfamiliarity of its taste in his mouth. He sat at the sticky bar counter, fingering the champagne glass and dreading returning home to a place and a mother whom he hated. She would be waiting for him at the door, rushing to open it and grabbing for his coat, not noticing the cold steel of his gaze or his curled lip with her one watery eye. No, he decided, he would not let her prodding questions or reaching fingers pull at his sanity any longer. Seventeen years of his life he had spent in isolation, turning his back on the woman who had ruined his schoolboy years and sent away the Daddy he had barely known. He was not going to suffer her homemade soups and hand-washed, warmly ironed laundry any more\ldots{}

The memories were too much for him. Igor stood up suddenly, spilling his spoiled drink over the stained floor and crushing the bag of anonymous cookies in an effort to jump away. His eyes were wild with the hatred his memories brewed in his mind; he stepped over the crumbs and puddles and rushed for the door of the cabin. Sliding the stiff, rusty portal open took great effort, and he nearly gave up. He threw his back against the wall and tried to catch his breath, the stale air straining his lungs. Why had he decided to put himself through this emotional torrent, all for a mother who had never loved him? His hands grasped at the crumpled paper in his pocket.

\texttt{We regret to inform you that your mother, Constance}

\texttt{Nelyubova, has passed away\ldots{}}

\noindent Igor breathed a little easier. At least the worst was over.

\texttt{She passed a few weeks ago of a failed heart. We}

\texttt{apologize that we could not get hold of you earlier to}

\texttt{tell you of the details of the funeral. She was buried}

 \texttt{last Sunday at Archangel Chamuel's Cemetery on the}

\texttt{corner of Seventh and Third Street\ldots{}}

\noindent A deep sigh sent dust swirling in little flurries around Igor's head. He coughed, a deep, throaty cough that racked his entire body.

At least he didn't have to look at her again. At least she was buried deep under the cold earth and all he needed was to throw away her belongings from his apartment. He was her only child, and he had been the inheritor---but of what? Of her old furniture that outdated even herself? Of his crinkled school photographs, worn from years of living in her apron pocket and being taken out by the hour, smoothed out, and kissed? Or maybe her refrigerator, which was sure to be stuffed full with her ceaseless cooking, neatly wrapped with his name on it, only to be tossed by his shameful hands in a metal bin behind the apartment building outside. He didn't want any of it\ldots{}

Igor held on to the railing as the train began to screech to a slow stop, the squeal of the rusty brakes piercing his ears. He grunted as he stumbled down the narrow, slippery steps and onto the cool platform. The air was crisp and clear; the sun radiant in a blue cloudless sky. Igor stuffed his cracked hands in his pockets and made his way through the eager crowd waiting to greet their loved ones from the train. Nobody waited for him. He walked through the station building, his heels clicking on the polished floors, casting a somber shadow over the cheery green and yellow pattern of the floor tiles.

He sat silent in the taxi. The driver was chatty and tried more than once to ask Igor about his plans for the weekend and tell him about a festival that would be taking place the next day. Igor ignored him; with each turn of the meter, he grew more and more uneasy at the thought of walking up to his mother's apartment door. Twice he almost asked the driver to stop and take him back to the much more welcoming train station, but his lips froze whenever he tried to form them around the words. The car pulled up in front of a vaguely familiar building; Igor paid the smiling driver and slammed the door.

His breath caught in his throat and his hands immediately turned clammy. He could feel his heart pounding, about to leap out of his chest. His cheeks burned as he slowly lifted his feet up the cracked stairs, worn with years of tread. He nodded to a pair of frowning old ladies he didn't recognize sitting on a dented bench by the rusty apartment building door that squeaked ominously as he pulled it open.

It had been years since he had last seen her. He remembered the day exactly---it had been a chilly day; he had been playing with his children in their living room, his wife amusedly watching them from her perch on the couch. Everybody had been laughing, yelling, tripping over each other and silently thanking God for each other. The knock came several times, for they did not hear it the first time over the great ruckus of merriment that burst through the seams of their play-time. Igor had jumped up off the floor and sprinted for the door, panting and laughing in the meantime. He thrust open the door and instantly the smile vanished as if by magic. He stared at his mother, standing on his porch and smiling weakly up at him. I wrote you a letter\ldots{}she had barely whispered. She smiled and reached to glance around him, to peek at the source of the squealing laughter coming from behind her son. Are those my grandchildren? She had asked him, peering up at him. I heard from your friends that you got married\ldots{} Igor couldn't help but stare at her right eyelid, half-expecting it to blink open along with her left one\ldots{} She stood in his doorway for minutes, gripping her one worn, beaten suitcase in her hands and gazing expectantly up at him with that one watery eye. His old feelings of disgust returned to him, churning in the pit of his stomach. Leave, he had spat at her. My children shall never be scarred by your disgraceful presence like I was my entire childhood. I don't want them to know you\ldots{}you're\ldots{}you're a witch. Leave! His words rolled out of his mouth and burned his tongue, filling his mouth with a vile taste like poison. He glared menacingly down at the woman whom he hated to call his mother until she set her face, turned her watery eye away from him and shuffled away from his door. She never looked back. Igor slammed the door and could not return to his family out of the concentrated rage that boiled his blood. The next week he received a letter in her handwriting; and the next, and the next. He never opened them, he never wrote back, but only sat, drinking whiskey and watching the white paper curl and fall to ashes in the pit of the fireplace.

Now he stood at her door. The creaky elevator doors screeched to a shut behind him, and the landing was suddenly cloaked in a ringing silence. His mother's neighbors, the senseless blokes who had sent that fateful telegram, had left a spare key on top of her apartment door frame. He felt like a traitor as he turned the brass key in the scratched lock and pushed open the door. His nose was invaded with the thick smell of roses which hung heavy in the air, no doubt from the funeral. Igor's conscience twitched. He forced himself to step into the old apartment, instantly feeling like a ten year old boy again. Nothing had changed. His pictures hung on the wall, stood on the dressers, peeked out of pockets of faded, ancient aprons. His mother's moth-eaten curtains hung in their familiar perch in the living room, pulled aside slightly, revealing a small sliver of the silver street that Igor used to watch as a child. There wasn't much to look at in the small one-bedroom apartment---his mother's bed was neatly made and his bed stood as if he had been living in it all this time. The headboard was neatly dusted and the pillows were carefully fluffed. Suddenly Igor felt guilt forming a heavy lump in his throat. His eyes scoured the apartment in which his mother had single-handedly raised him. A sealed envelope lay in the center of the scratched, stained kitchen table. Igor reached for it---it was addressed to him in the same script as the hundreds of others he had received over the years at his home in Gnill. His numb fingers carefully tore the envelope open and produced a short note neatly written in his mother's old handwriting.

His face froze with his jaw slightly ajar. He reread the note again, and again.

\textit{My dearest son Igor,}

\textit{I have always and will always love you, no matter how you may think of}

\textit{me. I wish that I had done more for you, somehow\ldots{}}

\noindent His eyes followed the words across the lines like a ping-pong ball in a heated game.

\textit{I'm sorry to have ruined your life by being such an invalid\ldots{}}

\noindent No. This couldn't be true.

\textit{I know how ashamed you must have felt growing up, with only a weak,}

\textit{one-eyed mother trying to fill the empty gap your father left when he}

\textit{abandoned us\ldots{}}

\noindent Yet the words were printed there on the page, despite his disbelief.

\textit{I meant to keep this hidden from you forever---no use in burdening you}

\textit{with information you didn't have the desire of knowing. But I feel that you}

\textit{should know this now. Love, when you were an infant, you suffered a head}

\textit{trauma\ldots{}}

\noindent Igor's leg gave out and he fell into the chair that seemed to have been pulled out just for him. 

\textit{You were on the verge of death. The doctors could do almost nothing. You}

\textit{lost your eye\ldots{}}

\noindent He grabbed his head with his hands. No! No! No! He sat gasping for breath, his face twisted in a look of horrific pain. He suddenly could not remember his mother's face; but he heard over and over again the echo of her warm, dejected voice at school, at home, on the street, calling out to him, I love you Igor, and he saw himself in his mind's eye as a small boy, heart filled to the brim with hurt, confusion, and hatred, pulling away from her reach.

\textit{I didn't want you to have to live your life as an invalid\ldots{}you were so}

\textit{young, barely even walking at the time\ldots{}}

\noindent Hot tears poured down Igor's flushed cheeks. How could he have lived so long, he cried over and over again, being so ignorant?

\textit{I love you so much my sunshine. Now you know I really will always be}

\textit{with you\ldots{}}

\noindent Igor stood up, wild with guilt and unbearable pain. Primitive, animalistic sobs erupted from his mouth and his body shook uncontrollably. Blinded by his tears, he stumbled into his mother's bedroom and fell face-first on her thin covers. ``I love you, mom\ldots{}'' he wailed. ``Mom I'm so sorry\ldots{}'' He hugged the loose pillow close to his face and cried like he used to in the school courtyard as a boy, hurt and alone. ``I love you mom,'' he sobbed, ``I love you so much\ldots{}'' 


\end{verseEnt}

\begin{verseEnt}{Hyperbole Love Letter}{Aurora Siira}
\lettrine{D}{ear} Waldo,

I would be telling you this in person, but since you've left me with the predicament of having to find you, you leave me no other choice than writing you this letter. Oh Waldo, I have no idea where to begin. Maybe it's best that I start from page one. You first caught my eye with your cute red and white striped shirt and hat, and those adorable glasses that suit your face perfectly. How could anyone miss that? Your dream of hiking across the world dazzled me, and I always looked forward to your postcards, and I didn't mind helping you look for your lost hiking gear either. It was like an epic scavenger hunt with you, with all the new obscure items to search for. However, now it just seems as if you're trying to hide from me all the time; whenever we go out, I end up losing you in the crowd, no matter if it's the beach or an amusement park. I don't understand anymore. I just get so frustrated constantly searching for you. So I've decided that it's finally come to the point for me to put the book down. It's over Waldo; I no longer desire to know where you are. I've reached the end of my list. You should find yourself.

\raggedleft{Sincerely, your lost love}
\end{verseEnt}

\begin{poetry}{Sensationalism}{Shree Sundaresh}
Uncomfortably wedged
behind the \large{\textbf{Decaying Carcass}} \normalsize
of a \Large{\textbf{Homicide,}} \normalsize
hastily hidden
within the shadows
of the \Large{\textbf{Nefarious Villain}} \normalsize
brought to light,
the grey letters of a trite article
sit soundlessly within the crease
of the newspaper,
slumped over with the weight of the
\Large{\textbf{Atrocities,}}
\bigskip
\Large{\textbf{Dehumanization,}} \normalsize
\bigskip
and
\LARGE{\textbf{Barbarity}} \normalsize
plaguing the world surrounding it.
\bigskip
Never to tell its humble story about how---\hspace{-0.6in} \huge{\textbf{Murder}}
\end{poetry}

\begin{poetry}{Needs Proper L\ ght\ ng}{Alyssa Thorne}
Another face.
Another name.
Another organ\ sm g\ v\ ng off heat.
Other than that?
Noth\ ng.
Nada,
Z\ p,
Z\ lch.
Same classes s\ nce k\ ndergarten?
Doesn't matter.
L\ ved down the street our ent\ re l\ ves?
Un\ mportant.
You st\ ll don't remember me.
\ 'm \ nconsequent\ al.
\ ns\ gn\ f\ cant.
\ nv\ s\ ble.
\bigskip
Am \  see through?
L\ ke a ghost?
A phantom?
No,
Those show wh\ spers.
H\ nts of ex\ stence.
Where are m\ ne?
You don't know.
\bigskip
Can no one take the t\ me,
Or make the effort?
Try,
To see.
\bigskip
A human being.
More than just wind.
Or matter floating in space.
\bigskip
Existent.
Definite.
\bigskip
\bigskip
Individual.
\end{poetry}

\begin{verseEnt}{The Last Good Thing about This Part of Town}{Jack Vann}
\lettrine{``P}{issah,''} he muttered to himself as he sat watching people busily flutter through the crowded streets of downtown Boston. It was a cold and dreary Saturday evening, the type of night in which many people prefer to be snuggled up by a fire reading a book or in the arms of a loved one. For Louie Miller, the frigid December air numbed the waves of resentment and self-pity that had become part of his daily routine. He took a swig from the brown bag tucked under his wool jacket and began to lose himself in his thoughts. 

It'd been months since he'd talked to her, but not a day had gone by when he hadn't thought about what had been.  He couldn't put his finger on exactly what had caused their falling out. Louie had become quick tempered and irritable; in all, unstable. Just the other week, he'd almost gotten into a fist fight at Cuff's because some muscle-head had bumped into him and made him spill the foam of his beer on the counter. Thinking about how happy they'd been together made his stomach churn. He hated himself for what he'd done.
  
Night after night, he tried to extinguish the fire inside with help from his ol' friend Jack, a feeble attempt to get his spirits up, to no avail. Louie knew that it wouldn't help, but it kept him from remembering all the piercing memories that plagued his thoughts every hour of the day. He'd wake up the next afternoon with a wicked headache and no recollection of the night before. 

Seeking another coping mechanism, he had tried to find a replacement partner downtown with the infamous ladies on the corner of Broad and Water, but their impassioned enchantment couldn't heal. He tried to fall in love with other women: taking them out to dinner, buying jewelry and champagne. But still, thoughts of her remained cemented in the back of his mind; he couldn't keep himself from seeing remnants of his past love.

Every day he sat on that low wall in Liberty Square looking at the bench across the street in front of Shaw's. There, they'd spent innumerable nights talking about silly things like their future, while sharing a plate of Chinese takeout in the brisk autumn air. He did that a lot; more so than what he felt was normal. He constantly relived moments from the past in places he encountered as he went through the motions of his daily routine. Sometimes he thought he did it just because he wanted to torture himself, a desire born from some twisted masochistic self-loathing.

No matter how hard he pushed himself, he couldn't keep himself from scanning the throngs of faces wherever he went. He knew he wouldn't find those glossy chestnut eyes, no matter how long he searched or how badly he wanted---needed---it. ``Have a good life,'' the note left on his front door had said.  Louie had assumed that she moved to California. ``She always said how much she'd love to move to California to sing,'' he remembered later that disastrous night in an alcohol induced malaise. Instinctively, his eyes sporadically jumped from person to person; frantically searching for a flash of that ebony cascade he was so adept at picking out of a crowd. For an instant, his heart jumped and a flood of burning anticipation washed over him, only to quickly be swept away.It was the same feeling that even after years of being together, he had gotten whenever their eyes had crossed paths.

His concentration was shattered by a thunderous round of applause from down the street.  He adjusted his beanie to shelter his ears from the frigid New England air, examined the gray palette of downtown Boston and trudged towards Kilby Street to see what was going on. He reached the corner, finding a couple hugging amorously amidst a crowd of onlookers with facial expressions akin to those when first seeing a baby. Louie knew what this was. He'd played it out countless times in his head, what it would have been like. He had run through what he needed to say, acted out the scene, but he still felt that he would forget that simple question.  The thought disgusted him; the thought of the words themselves made the hair on his arms stand on edge. ``If only they knew\ldots{}'' he murmured bitterly as he turned away. He quickly wiped his eyes, blaming the glacial drafts for his teary-eyed complexion, as he headed back into a boundless sea of all-too-vivid memories. Beneath that fa�ade of cold-heartedness and self-destructive decisions, the remnants of a steadfast love still remained, waiting to be rekindled.
\end{verseEnt}

\begin{verseEnt}{Shadows}{John Wang-Hu}
\lettrine{T}{here} was no light for me again. The monsters slowly crept into the room and swallowed it whole. The darkness embraced my conscious gently and I responded to it idle. They greedily cut off my light when I didn't make ends meet. I quickly fumbled for my lighter, releasing the withheld breath once the radius encompassed me, fending off the cold. I took to the process of giving life to the single bedroom apartment until it glowed ablaze. The room looked cluttered but in reality, it was strewn with wickless candles. Over a period of time, I learned how to shuffle my way through the sticks, avoiding the slip. I slowly crossed the floor like a penguin and climbed back under the covers and stared up at the ceiling.

It wasn't until the next morning when it dawned on me: I needed to step out and buy the week's groceries. I put on the weathered earthen raincoat and boots and stepped out into the rain. I trudged on the lighter side of the road, where the potholes scarred the face of civilization. The sky was still the usual mucky gray with a few dark clouds hanging over the people's thoughts. I headed west toward the local flea market, glancing backwards over my shoulders every few minutes. 

I arrived at an unassuming stand, registered under the vague name of White. Ironically, the White only sold brown rice and bread, of which I bought a few pounds of rice and loaves of bread. As I was leaving, my ears caught hold of loud applause and whistles around the corner. I curiously walked towards the crowd and found myself staring at a live circus performance amidst the raining weather. From one glance, I could have told that the clown was from the east. He wasn't from these parts of town because he was dressed in color, with his rainbow striped suit and his genuine smile. I looked around at the audience, composed of children and adults in their forties. The adults held the hands of their children, leaning on one leg as they looked with their painted smiles. I felt impressed by the fearless display the clown was giving. The clown balanced on a wet ball whilst juggling several sharp blades without hesitation. The blades twirled in the darkening air, reflecting the lifeless eyes of the entire crowd. I eventually drew away from the crowd as an innate uncomfortable depression that started to settle over my thoughts. 

I quickened my pace as I noticed that night was coming. It was natural to me; I've never given thought to my hurrying home. Maybe it was perhaps I was distracted by the clown but I didn't make it home in time. The darkness fell below the horizon and the nearest street lamp was in the next city over. I took out my lighter and again, took myself out of the world. I moved as slowly as I could now, cupping the small flame enough that it still shone through death's hands. The light from an up ahead house cast strange shadows of upcoming passersby. I held my gaze low as I passed the convoluted shadows, avoiding the touch of my shadow with theirs. An unsettling chill settled subtly upon my spine and I shuddered as the shadows of the past haunted me again.

I trembled violently with fear, as the thoughts evaded my subconscious; I smelled again, the burnt smell of wood, old worn out linen, and flesh. The dead corpses of my parents found under the rubble, blacked with soot and ashen with crust. I can feel the intangible hug of their ghosts, as they wrap their bones around from the back, strangling me from behind. Their charcoal bones, empty screams, and hollow sockets lay hidden somewhere in the darkness. I only wanted to have freedom, but in exchange\ldots{}

I neared the corner to my apartment, but as I climbed the stairs, the lighter gave out. I flicked furiously several times, but to my horror, only sparks flew across my thumb. I stood there paralyzed, as I waited there for a cold breath to roll down my neck. Nothing came for another few minutes, I felt strangely disappointed. Was it perhaps the clown I saw earlier and his daring acts or have I met my end? The darkness seemed to lighten in density and I began to breath normally again. I stood there thawing from the paralysis, holding onto the railing and feeling my way upwards until I heard a sharp hiss on top of the staircase. I rigidly shoved my hands into my pockets and pulled out a matchbox, striking the head once, twice, three times. 

\end{verseEnt}

\begin{poetry}{Somewhere Else}{Holly Weinrauch}
Up high with you,
Above mountains
(Although we knew there would not be any).
\bigskip
I still wanted to see your face,
When we were soaring atop
Houses and lakes and pretty little girls with blushing balloons.
I wanted that one chance with you,
To see everything that we should have.
\bigskip
You said that we could get there early and be awestruck
When the primary colors combusted against the pink atmosphere
And sit in dew covered grass
And ardently wait our turn.
\bigskip
But you went up before me,
And I really don't mind
I was never fond of heights anyway.
\end{poetry}

\begin{poetry}{Blah, Blah, Blah}{Nabeel Zafrullah}
The identical words follow each other forever,
Like a boy follows his friend,
Like a girl follows Hannah Montana,
Like the moon follows the sunset;
\bigskip
Contingency of the same leads to boredom,
So why follow a person like ``blah following `blah'?''
Be different,
Like the stripes on every zebra,
Like every person's fingerprint.
\end{poetry}


\end{document}
